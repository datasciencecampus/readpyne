%% Generated by Sphinx.
\def\sphinxdocclass{report}
\documentclass[letterpaper,10pt,english]{sphinxmanual}
\ifdefined\pdfpxdimen
   \let\sphinxpxdimen\pdfpxdimen\else\newdimen\sphinxpxdimen
\fi \sphinxpxdimen=.75bp\relax

\PassOptionsToPackage{warn}{textcomp}
\usepackage[utf8]{inputenc}
\ifdefined\DeclareUnicodeCharacter
% support both utf8 and utf8x syntaxes
  \ifdefined\DeclareUnicodeCharacterAsOptional
    \def\sphinxDUC#1{\DeclareUnicodeCharacter{"#1}}
  \else
    \let\sphinxDUC\DeclareUnicodeCharacter
  \fi
  \sphinxDUC{00A0}{\nobreakspace}
  \sphinxDUC{2500}{\sphinxunichar{2500}}
  \sphinxDUC{2502}{\sphinxunichar{2502}}
  \sphinxDUC{2514}{\sphinxunichar{2514}}
  \sphinxDUC{251C}{\sphinxunichar{251C}}
  \sphinxDUC{2572}{\textbackslash}
\fi
\usepackage{cmap}
\usepackage[T1]{fontenc}
\usepackage{amsmath,amssymb,amstext}
\usepackage{babel}



\usepackage{times}
\expandafter\ifx\csname T@LGR\endcsname\relax
\else
% LGR was declared as font encoding
  \substitutefont{LGR}{\rmdefault}{cmr}
  \substitutefont{LGR}{\sfdefault}{cmss}
  \substitutefont{LGR}{\ttdefault}{cmtt}
\fi
\expandafter\ifx\csname T@X2\endcsname\relax
  \expandafter\ifx\csname T@T2A\endcsname\relax
  \else
  % T2A was declared as font encoding
    \substitutefont{T2A}{\rmdefault}{cmr}
    \substitutefont{T2A}{\sfdefault}{cmss}
    \substitutefont{T2A}{\ttdefault}{cmtt}
  \fi
\else
% X2 was declared as font encoding
  \substitutefont{X2}{\rmdefault}{cmr}
  \substitutefont{X2}{\sfdefault}{cmss}
  \substitutefont{X2}{\ttdefault}{cmtt}
\fi


\usepackage[Bjarne]{fncychap}
\usepackage{sphinx}

\fvset{fontsize=\small}
\usepackage{geometry}

% Include hyperref last.
\usepackage{hyperref}
% Fix anchor placement for figures with captions.
\usepackage{hypcap}% it must be loaded after hyperref.
% Set up styles of URL: it should be placed after hyperref.
\urlstyle{same}
\addto\captionsenglish{\renewcommand{\contentsname}{Contents:}}

\usepackage{sphinxmessages}
\setcounter{tocdepth}{1}



\title{readpyne}
\date{Apr 17, 2019}
\release{}
\author{Art Eidukas @ Data Science Campus}
\newcommand{\sphinxlogo}{\vbox{}}
\renewcommand{\releasename}{}
\makeindex
\begin{document}

\pagestyle{empty}
\sphinxmaketitle
\pagestyle{plain}
\sphinxtableofcontents
\pagestyle{normal}
\phantomsection\label{\detokenize{index::doc}}



\chapter{API Reference}
\label{\detokenize{api:module-readpyne.core}}\label{\detokenize{api:api-reference}}\label{\detokenize{api::doc}}\index{readpyne.core (module)@\spxentry{readpyne.core}\spxextra{module}}

\section{readpyne.core}
\label{\detokenize{api:readpyne-core}}
the core functionality for readpyne.
\index{binit() (in module readpyne.core)@\spxentry{binit()}\spxextra{in module readpyne.core}}

\begin{fulllineitems}
\phantomsection\label{\detokenize{api:readpyne.core.binit}}\pysiglinewithargsret{\sphinxcode{\sphinxupquote{readpyne.core.}}\sphinxbfcode{\sphinxupquote{binit}}}{\emph{l\_arrays}, \emph{n\_features=100}}{}
Takes in a list of 3 1d arrays of length \sphinxcode{\sphinxupquote{n}} and firstly it bins it
into a set number of features dictated by \sphinxcode{\sphinxupquote{n\_features}}. This produces
an array of length \sphinxcode{\sphinxupquote{n\_features}}. Then it stacks the three arrays
into a 1d array of length \sphinxcode{\sphinxupquote{3 * n\_features}}

\begin{sphinxadmonition}{note}{Note:}
This function does not check if the len of the input list of arrays is 3.
\end{sphinxadmonition}
\begin{quote}\begin{description}
\item[{Parameters}] \leavevmode\begin{itemize}
\item {} 
\sphinxstyleliteralstrong{\sphinxupquote{l\_arrays}} (\sphinxstyleliteralemphasis{\sphinxupquote{list}}) \textendash{} A list of arrays of length \sphinxcode{\sphinxupquote{n}}. Usually this will be a vertically
collapsed image that has been passed through \sphinxcode{\sphinxupquote{cv2.split}} to split its
channels.

\item {} 
\sphinxstyleliteralstrong{\sphinxupquote{n\_features}} (\sphinxstyleliteralemphasis{\sphinxupquote{int}}) \textendash{} An integer telling the function how many features to produce per array.
This dictates the shape of end array as the resulting array will have a
length of \sphinxcode{\sphinxupquote{3 * n\_features}}

\end{itemize}

\item[{Returns}] \leavevmode
A numpy array of length \sphinxcode{\sphinxupquote{3 * n\_features}}

\item[{Return type}] \leavevmode
numpy.array

\end{description}\end{quote}

\end{fulllineitems}

\index{blobify() (in module readpyne.core)@\spxentry{blobify()}\spxextra{in module readpyne.core}}

\begin{fulllineitems}
\phantomsection\label{\detokenize{api:readpyne.core.blobify}}\pysiglinewithargsret{\sphinxcode{\sphinxupquote{readpyne.core.}}\sphinxbfcode{\sphinxupquote{blobify}}}{\emph{img}}{}
Take in an image and pass it through \sphinxcode{\sphinxupquote{cv2.dnn.blobFromImage}}
\begin{quote}\begin{description}
\item[{Parameters}] \leavevmode
\sphinxstyleliteralstrong{\sphinxupquote{blob}} (\sphinxstyleliteralemphasis{\sphinxupquote{cv2.blob}}) \textendash{} An image that has been \sphinxcode{\sphinxupquote{blobified}} by cv2.
(see the blobify function documentation)

\item[{Returns}] \leavevmode


\item[{Return type}] \leavevmode
cv2.blob

\end{description}\end{quote}

\end{fulllineitems}

\index{boxes() (in module readpyne.core)@\spxentry{boxes()}\spxextra{in module readpyne.core}}

\begin{fulllineitems}
\phantomsection\label{\detokenize{api:readpyne.core.boxes}}\pysiglinewithargsret{\sphinxcode{\sphinxupquote{readpyne.core.}}\sphinxbfcode{\sphinxupquote{boxes}}}{\emph{img}}{}
Take in an image, resize it, predict boxes. Then perform
expansion of the boxes to the width of the receipt and then
perform \sphinxcode{\sphinxupquote{non\_max\_supression}}.
\begin{quote}\begin{description}
\item[{Parameters}] \leavevmode
\sphinxstyleliteralstrong{\sphinxupquote{img}} (\sphinxstyleliteralemphasis{\sphinxupquote{numpy.array}}) \textendash{} A numpy array representation of an image.

\item[{Returns}] \leavevmode
\begin{itemize}
\item {} 
\sphinxstyleemphasis{np.array} \textendash{} The original image

\item {} 
\sphinxstyleemphasis{np.array} \textendash{} Predicted subsets for the image.

\end{itemize}


\end{description}\end{quote}

\end{fulllineitems}

\index{decode() (in module readpyne.core)@\spxentry{decode()}\spxextra{in module readpyne.core}}

\begin{fulllineitems}
\phantomsection\label{\detokenize{api:readpyne.core.decode}}\pysiglinewithargsret{\sphinxcode{\sphinxupquote{readpyne.core.}}\sphinxbfcode{\sphinxupquote{decode}}}{\emph{scores}, \emph{geo}}{}
This takes the geometries and confidence scores and produces bounding box values.
The inputs to this function come from the EAST model with 2 layers.

\begin{sphinxadmonition}{note}{Note:}
This function borrows heavily from:
\sphinxurl{https://www.pyimagesearch.com/2018/08/20/opencv-text-detection-east-text-detector/}
\end{sphinxadmonition}
\begin{quote}\begin{description}
\item[{Parameters}] \leavevmode\begin{itemize}
\item {} 
\sphinxstyleliteralstrong{\sphinxupquote{scores}} (\sphinxstyleliteralemphasis{\sphinxupquote{numpy.array}}) \textendash{} A numpy array of size \sphinxcode{\sphinxupquote{(number of found boxes, )}} indicating the assigned
confidence scores for each box.

\item {} 
\sphinxstyleliteralstrong{\sphinxupquote{geo}} (\sphinxstyleliteralemphasis{\sphinxupquote{numpy.array}}) \textendash{} A numpy array of size \sphinxcode{\sphinxupquote{(number of boxes, 5)}} coming out from EAST
that describes where the boxes are.

\end{itemize}

\item[{Returns}] \leavevmode
A numpy array of shape \sphinxcode{\sphinxupquote{(n, 5)}} containing the confidences and box locations
for the boxes that are of a certain confidence.

\item[{Return type}] \leavevmode
numpy.array

\end{description}\end{quote}

\end{fulllineitems}

\index{expand() (in module readpyne.core)@\spxentry{expand()}\spxextra{in module readpyne.core}}

\begin{fulllineitems}
\phantomsection\label{\detokenize{api:readpyne.core.expand}}\pysiglinewithargsret{\sphinxcode{\sphinxupquote{readpyne.core.}}\sphinxbfcode{\sphinxupquote{expand}}}{\emph{arr}, \emph{shape}}{}
Function to expand an array of image coordinates to the full width of
the image.
\begin{quote}\begin{description}
\item[{Parameters}] \leavevmode\begin{itemize}
\item {} 
\sphinxstyleliteralstrong{\sphinxupquote{arr}} (\sphinxstyleliteralemphasis{\sphinxupquote{numpy.array}}) \textendash{} A two dimensional array of coordinates which are in the form of
\sphinxcode{\sphinxupquote{startX, startY, endX, endY}}

\item {} 
\sphinxstyleliteralstrong{\sphinxupquote{shape}} (\sphinxstyleliteralemphasis{\sphinxupquote{tuple}}) \textendash{} A tuple containing the shape of the original receipt image.
Easily accessible through the use of the \sphinxcode{\sphinxupquote{.shape}} method on an
image.

\end{itemize}

\item[{Returns}] \leavevmode
A numpy array of the same shape as the original array but with
the \sphinxcode{\sphinxupquote{x}} values in each row expanded to the width of the image.

\item[{Return type}] \leavevmode
numpy.array

\end{description}\end{quote}

\end{fulllineitems}

\index{features() (in module readpyne.core)@\spxentry{features()}\spxextra{in module readpyne.core}}

\begin{fulllineitems}
\phantomsection\label{\detokenize{api:readpyne.core.features}}\pysiglinewithargsret{\sphinxcode{\sphinxupquote{readpyne.core.}}\sphinxbfcode{\sphinxupquote{features}}}{\emph{img}, \emph{subsets}}{}
Take an image and its subsets created from \sphinxcode{\sphinxupquote{boxes}} and
produce histogram based features for each subset.
\begin{quote}\begin{description}
\item[{Parameters}] \leavevmode\begin{itemize}
\item {} 
\sphinxstyleliteralstrong{\sphinxupquote{img}} (\sphinxstyleliteralemphasis{\sphinxupquote{numpy.array}}) \textendash{} A numpy array representation of an image.

\item {} 
\sphinxstyleliteralstrong{\sphinxupquote{subsets}} (\sphinxstyleliteralemphasis{\sphinxupquote{list}}) \textendash{} List of numpy arrays of the subsets.

\end{itemize}

\item[{Returns}] \leavevmode
\begin{itemize}
\item {} 
\sphinxstyleemphasis{np.array} \textendash{} The original image

\item {} 
\sphinxstyleemphasis{list} \textendash{} A list of 1d numpy arrays

\end{itemize}


\end{description}\end{quote}

\end{fulllineitems}

\index{forward() (in module readpyne.core)@\spxentry{forward()}\spxextra{in module readpyne.core}}

\begin{fulllineitems}
\phantomsection\label{\detokenize{api:readpyne.core.forward}}\pysiglinewithargsret{\sphinxcode{\sphinxupquote{readpyne.core.}}\sphinxbfcode{\sphinxupquote{forward}}}{\emph{blob}}{}
Take in a \sphinxcode{\sphinxupquote{cv2 blob}} and pass it forward through an \sphinxcode{\sphinxupquote{EAST}} model.

\begin{sphinxadmonition}{note}{Note:}
The layers in the model by default are \sphinxcode{\sphinxupquote{"feature\_fusion/Conv\_7/Sigmoid", "feature\_fusion/concat\_3"}}
\end{sphinxadmonition}
\begin{quote}\begin{description}
\item[{Parameters}] \leavevmode
\sphinxstyleliteralstrong{\sphinxupquote{blob}} (\sphinxstyleliteralemphasis{\sphinxupquote{cv2.blob}}) \textendash{} An image that has been \sphinxcode{\sphinxupquote{blobified}} by cv2.
(see the blobify function documentation)

\item[{Returns}] \leavevmode
\begin{itemize}
\item {} 
\sphinxstyleemphasis{numpy.array} \textendash{} Scores array for each box found.

\item {} 
\sphinxstyleemphasis{numpy.array} \textendash{} Bounding box locations from EAST

\end{itemize}


\end{description}\end{quote}

\end{fulllineitems}

\index{get\_subsets() (in module readpyne.core)@\spxentry{get\_subsets()}\spxextra{in module readpyne.core}}

\begin{fulllineitems}
\phantomsection\label{\detokenize{api:readpyne.core.get_subsets}}\pysiglinewithargsret{\sphinxcode{\sphinxupquote{readpyne.core.}}\sphinxbfcode{\sphinxupquote{get\_subsets}}}{\emph{img}, \emph{boxes}}{}
Take an image and box locations. Then cut out these boxes from the given image.
\begin{quote}\begin{description}
\item[{Parameters}] \leavevmode\begin{itemize}
\item {} 
\sphinxstyleliteralstrong{\sphinxupquote{img}} (\sphinxstyleliteralemphasis{\sphinxupquote{numpy.array}}) \textendash{} A numpy array representation of an image.

\item {} 
\sphinxstyleliteralstrong{\sphinxupquote{boxes}} (\sphinxstyleliteralemphasis{\sphinxupquote{iterable}}) \textendash{} An iterable (most likely a numpy.array) containing box coordinates.

\end{itemize}

\item[{Returns}] \leavevmode
A list of subsets.

\item[{Return type}] \leavevmode
list

\end{description}\end{quote}

\end{fulllineitems}

\index{hist() (in module readpyne.core)@\spxentry{hist()}\spxextra{in module readpyne.core}}

\begin{fulllineitems}
\phantomsection\label{\detokenize{api:readpyne.core.hist}}\pysiglinewithargsret{\sphinxcode{\sphinxupquote{readpyne.core.}}\sphinxbfcode{\sphinxupquote{hist}}}{\emph{img}}{}
Histogram creation function. This function takes an input image and then
collapses it vertically by using \sphinxcode{\sphinxupquote{np.mean}}. It does this for each channel
of the image as it uses \sphinxcode{\sphinxupquote{cv2.split}} to get each channel.
\begin{quote}\begin{description}
\item[{Parameters}] \leavevmode
\sphinxstyleliteralstrong{\sphinxupquote{img}} (\sphinxstyleliteralemphasis{\sphinxupquote{numpy.array}}) \textendash{} This is a numpy array representation of an input image. Expected shape
for this array is \sphinxcode{\sphinxupquote{(n,m,3)}} where \sphinxcode{\sphinxupquote{n}} is the height and \sphinxcode{\sphinxupquote{m}} is the
width of the image.

\item[{Returns}] \leavevmode
A list of numpy arrays. Each array will be of length \sphinxcode{\sphinxupquote{m}} where
\sphinxcode{\sphinxupquote{m}} is the width of the input image.

\item[{Return type}] \leavevmode
list

\end{description}\end{quote}

\end{fulllineitems}

\index{process() (in module readpyne.core)@\spxentry{process()}\spxextra{in module readpyne.core}}

\begin{fulllineitems}
\phantomsection\label{\detokenize{api:readpyne.core.process}}\pysiglinewithargsret{\sphinxcode{\sphinxupquote{readpyne.core.}}\sphinxbfcode{\sphinxupquote{process}}}{\emph{img}}{}
Function that performs preprocessing before histograms are created.
\begin{quote}\begin{description}
\item[{Parameters}] \leavevmode
\sphinxstyleliteralstrong{\sphinxupquote{img}} (\sphinxstyleliteralemphasis{\sphinxupquote{numpy.array}}) \textendash{} A numpy array of shape (n,m,3) containing an image.
Usually this will be a subset of a larger receipt.

\item[{Returns}] \leavevmode
A numpy array representation of the input image after
all the processing was applied.

\item[{Return type}] \leavevmode
numpy.array

\end{description}\end{quote}

\end{fulllineitems}

\index{resize() (in module readpyne.core)@\spxentry{resize()}\spxextra{in module readpyne.core}}

\begin{fulllineitems}
\phantomsection\label{\detokenize{api:readpyne.core.resize}}\pysiglinewithargsret{\sphinxcode{\sphinxupquote{readpyne.core.}}\sphinxbfcode{\sphinxupquote{resize}}}{\emph{img}}{}
This function resizes an input image to the correct dimensions for the
\sphinxcode{\sphinxupquote{EAST}} model that is used for text detection.

The dimensions for \sphinxcode{\sphinxupquote{EAST}} have to be divisable by 32. This function pads
the bottom and the right side of the image by a as many pixels as it needs
for this to happen.

\begin{sphinxadmonition}{note}{Note:}
The image is padded with white as the color and this is then propogated to
the rest of the pipeline under normal circumstances.
\end{sphinxadmonition}
\begin{quote}\begin{description}
\item[{Parameters}] \leavevmode
\sphinxstyleliteralstrong{\sphinxupquote{img}} (\sphinxstyleliteralemphasis{\sphinxupquote{numpy.array}}) \textendash{} This is a numpy array representation of an input image. Expected shape
for this array is \sphinxcode{\sphinxupquote{(n,m,3)}} where \sphinxcode{\sphinxupquote{n}} is the height and \sphinxcode{\sphinxupquote{m}} is the
width of the image.

\item[{Returns}] \leavevmode
\sphinxstylestrong{img} \textendash{} The white padded image with dimensions now divisable by 32.

\item[{Return type}] \leavevmode
numpy.array

\end{description}\end{quote}

\end{fulllineitems}

\index{stack() (in module readpyne.core)@\spxentry{stack()}\spxextra{in module readpyne.core}}

\begin{fulllineitems}
\phantomsection\label{\detokenize{api:readpyne.core.stack}}\pysiglinewithargsret{\sphinxcode{\sphinxupquote{readpyne.core.}}\sphinxbfcode{\sphinxupquote{stack}}}{\emph{features}}{}
Stack features. Basically take in a list containing
tuples of subsets and histogram based features and
stack them all up.
\begin{quote}\begin{description}
\item[{Parameters}] \leavevmode
\sphinxstyleliteralstrong{\sphinxupquote{features}} (\sphinxstyleliteralemphasis{\sphinxupquote{list}}) \textendash{} List of type \sphinxcode{\sphinxupquote{{[}({[}subsets{]}, {[}features{]}), ...{]}}}

\item[{Returns}] \leavevmode
List of type \sphinxcode{\sphinxupquote{{[}{[}all\_subsets{]},{[}all\_features{]}{]}}}

\item[{Return type}] \leavevmode
list

\end{description}\end{quote}

\end{fulllineitems}

\phantomsection\label{\detokenize{api:module-readpyne.decorators}}\index{readpyne.decorators (module)@\spxentry{readpyne.decorators}\spxextra{module}}

\section{readpyne.decorators}
\label{\detokenize{api:readpyne-decorators}}
A few helpful decorators for the rest of the code.
\index{timeit() (in module readpyne.decorators)@\spxentry{timeit()}\spxextra{in module readpyne.decorators}}

\begin{fulllineitems}
\phantomsection\label{\detokenize{api:readpyne.decorators.timeit}}\pysiglinewithargsret{\sphinxcode{\sphinxupquote{readpyne.decorators.}}\sphinxbfcode{\sphinxupquote{timeit}}}{\emph{fn}}{}
A simple time decorator. Its not as cool as timeit.
Only runs once.
\begin{quote}\begin{description}
\item[{Parameters}] \leavevmode
\sphinxstyleliteralstrong{\sphinxupquote{fn}} (\sphinxstyleliteralemphasis{\sphinxupquote{function}}) \textendash{} A function to be timed

\item[{Returns}] \leavevmode


\item[{Return type}] \leavevmode
function

\end{description}\end{quote}

\end{fulllineitems}

\index{unfold\_args() (in module readpyne.decorators)@\spxentry{unfold\_args()}\spxextra{in module readpyne.decorators}}

\begin{fulllineitems}
\phantomsection\label{\detokenize{api:readpyne.decorators.unfold_args}}\pysiglinewithargsret{\sphinxcode{\sphinxupquote{readpyne.decorators.}}\sphinxbfcode{\sphinxupquote{unfold\_args}}}{\emph{fn}}{}
Take in a function that takes in positional arguments
and makes it so you can pass in an unfoldable iterable as
the singular argument which then gets unfolded into the
positional arguments.
\begin{quote}\begin{description}
\item[{Parameters}] \leavevmode
\sphinxstyleliteralstrong{\sphinxupquote{fn}} (\sphinxstyleliteralemphasis{\sphinxupquote{function}}) \textendash{} 

\item[{Returns}] \leavevmode


\item[{Return type}] \leavevmode
function

\end{description}\end{quote}

\end{fulllineitems}

\phantomsection\label{\detokenize{api:module-readpyne.model}}\index{readpyne.model (module)@\spxentry{readpyne.model}\spxextra{module}}

\section{readpyne.model}
\label{\detokenize{api:readpyne-model}}
Model training, prediction and data making functions
\index{extract() (in module readpyne.model)@\spxentry{extract()}\spxextra{in module readpyne.model}}

\begin{fulllineitems}
\phantomsection\label{\detokenize{api:readpyne.model.extract}}\pysiglinewithargsret{\sphinxcode{\sphinxupquote{readpyne.model.}}\sphinxbfcode{\sphinxupquote{extract}}}{\emph{input\_folder}, \emph{classifier}, \emph{output\_folder=None}}{}
A function that uses a trained sklearn classifier to extract the lines
\begin{quote}\begin{description}
\item[{Parameters}] \leavevmode\begin{itemize}
\item {} 
\sphinxstyleliteralstrong{\sphinxupquote{input\_folder}} (\sphinxstyleliteralemphasis{\sphinxupquote{str}}) \textendash{} folder path to images

\item {} 
\sphinxstyleliteralstrong{\sphinxupquote{classifier}} (\sphinxstyleliteralemphasis{\sphinxupquote{sklearn model}}) \textendash{} sklearn model for classification

\item {} 
\sphinxstyleliteralstrong{\sphinxupquote{output\_folder}} (\sphinxstyleliteralemphasis{\sphinxupquote{str}}) \textendash{} if provided will save the predicted lines

\end{itemize}

\item[{Returns}] \leavevmode
a list of cutout lines in numpy array form

\item[{Return type}] \leavevmode
list

\end{description}\end{quote}

\end{fulllineitems}

\index{make\_training\_data() (in module readpyne.model)@\spxentry{make\_training\_data()}\spxextra{in module readpyne.model}}

\begin{fulllineitems}
\phantomsection\label{\detokenize{api:readpyne.model.make_training_data}}\pysiglinewithargsret{\sphinxcode{\sphinxupquote{readpyne.model.}}\sphinxbfcode{\sphinxupquote{make\_training\_data}}}{\emph{input\_folder}, \emph{image\_re}, \emph{output\_folder=None}}{}
Make training data from a folder of images.
\begin{quote}\begin{description}
\item[{Parameters}] \leavevmode\begin{itemize}
\item {} 
\sphinxstyleliteralstrong{\sphinxupquote{input\_folder}} (\sphinxstyleliteralemphasis{\sphinxupquote{str}}) \textendash{} Folder where the images are stored

\item {} 
\sphinxstyleliteralstrong{\sphinxupquote{image\_re}} (\sphinxstyleliteralemphasis{\sphinxupquote{str}}) \textendash{} Regular expression string that will filter only the images
in the folder.

\item {} 
\sphinxstyleliteralstrong{\sphinxupquote{output\_folder}} (\sphinxstyleliteralemphasis{\sphinxupquote{str}}) \textendash{} Folder where will the data will be saved. If not provided then
data won’t be saved and will be just returned.

\end{itemize}

\item[{Returns}] \leavevmode
\begin{itemize}
\item {} 
\sphinxstyleemphasis{list} \textendash{} a set of subsets from the images

\item {} 
\sphinxstyleemphasis{list} \textendash{} the stacked image features

\end{itemize}


\end{description}\end{quote}

\end{fulllineitems}

\index{status() (in module readpyne.model)@\spxentry{status()}\spxextra{in module readpyne.model}}

\begin{fulllineitems}
\phantomsection\label{\detokenize{api:readpyne.model.status}}\pysiglinewithargsret{\sphinxcode{\sphinxupquote{readpyne.model.}}\sphinxbfcode{\sphinxupquote{status}}}{\emph{name}, \emph{x}, \emph{y}, \emph{model}}{}
This function is responsible for reporting the quality of the model.
\begin{quote}\begin{description}
\item[{Parameters}] \leavevmode\begin{itemize}
\item {} 
\sphinxstyleliteralstrong{\sphinxupquote{name}} (\sphinxstyleliteralemphasis{\sphinxupquote{str}}) \textendash{} A string that will be the title of report

\item {} 
\sphinxstyleliteralstrong{\sphinxupquote{x}} (\sphinxstyleliteralemphasis{\sphinxupquote{numpy.array}}) \textendash{} A numpy array with training features.

\item {} 
\sphinxstyleliteralstrong{\sphinxupquote{y}} (\sphinxstyleliteralemphasis{\sphinxupquote{numpy.array}}) \textendash{} A numpy array with labels

\item {} 
\sphinxstyleliteralstrong{\sphinxupquote{model}} (\sphinxstyleliteralemphasis{\sphinxupquote{sklearn model}}) \textendash{} A model to be scored

\end{itemize}

\item[{Returns}] \leavevmode


\item[{Return type}] \leavevmode
None

\end{description}\end{quote}

\end{fulllineitems}

\index{train\_model() (in module readpyne.model)@\spxentry{train\_model()}\spxextra{in module readpyne.model}}

\begin{fulllineitems}
\phantomsection\label{\detokenize{api:readpyne.model.train_model}}\pysiglinewithargsret{\sphinxcode{\sphinxupquote{readpyne.model.}}\sphinxbfcode{\sphinxupquote{train\_model}}}{\emph{X}, \emph{y}, \emph{report=False}, \emph{save\_path=None}, \emph{frac\_test=0.25}, \emph{sk\_model=\textless{}class 'sklearn.neighbors.classification.KNeighborsClassifier'\textgreater{}}, \emph{model\_params=\{'n\_neighbors': 2\}}}{}
Given a set of data an labels. Train a sklearn model.
\begin{quote}\begin{description}
\item[{Parameters}] \leavevmode\begin{itemize}
\item {} 
\sphinxstyleliteralstrong{\sphinxupquote{X}} (\sphinxstyleliteralemphasis{\sphinxupquote{numpy.array}}) \textendash{} A numpy array with training features.

\item {} 
\sphinxstyleliteralstrong{\sphinxupquote{y}} (\sphinxstyleliteralemphasis{\sphinxupquote{numpy.array}}) \textendash{} A numpy array with labels.

\item {} 
\sphinxstyleliteralstrong{\sphinxupquote{report}} (\sphinxstyleliteralemphasis{\sphinxupquote{bool}}) \textendash{} A boolean that tells you if you need the reporting procedure to run.

\item {} 
\sphinxstyleliteralstrong{\sphinxupquote{save\_path}} (\sphinxstyleliteralemphasis{\sphinxupquote{str}}) \textendash{} A path to save the model to

\item {} 
\sphinxstyleliteralstrong{\sphinxupquote{frac\_test}} (\sphinxstyleliteralemphasis{\sphinxupquote{float}}) \textendash{} A float indicating the amount of data to keep for testing.

\item {} 
\sphinxstyleliteralstrong{\sphinxupquote{sk\_model}} (\sphinxstyleliteralemphasis{\sphinxupquote{sklearn model object}}) \textendash{} sklearn model that will be trained. An instance of it will be created and then
trained.

\item {} 
\sphinxstyleliteralstrong{\sphinxupquote{model\_params}} (\sphinxstyleliteralemphasis{\sphinxupquote{dict}}) \textendash{} A dict of parameters to be passed to the sklearn model.

\end{itemize}

\item[{Returns}] \leavevmode
\begin{itemize}
\item {} 
\sphinxstyleemphasis{sklearn model} \textendash{} Trained sklearn model

\item {} 
\sphinxstyleemphasis{tuple} \textendash{} A tuple containing the untouched test data \sphinxcode{\sphinxupquote{(X\_test, y\_test)}}

\end{itemize}


\end{description}\end{quote}

\end{fulllineitems}

\phantomsection\label{\detokenize{api:module-readpyne.io}}\index{readpyne.io (module)@\spxentry{readpyne.io}\spxextra{module}}

\section{readpyne.io}
\label{\detokenize{api:readpyne-io}}
All input output functions
\index{cutout\_save() (in module readpyne.io)@\spxentry{cutout\_save()}\spxextra{in module readpyne.io}}

\begin{fulllineitems}
\phantomsection\label{\detokenize{api:readpyne.io.cutout_save}}\pysiglinewithargsret{\sphinxcode{\sphinxupquote{readpyne.io.}}\sphinxbfcode{\sphinxupquote{cutout\_save}}}{\emph{path}, \emph{img}, \emph{subsets}}{}
Take an image and its subsets and export it to the given path.
\begin{quote}\begin{description}
\item[{Parameters}] \leavevmode\begin{itemize}
\item {} 
\sphinxstyleliteralstrong{\sphinxupquote{path}} (\sphinxstyleliteralemphasis{\sphinxupquote{str}}) \textendash{} A path to be used to save all subsets.

\item {} 
\sphinxstyleliteralstrong{\sphinxupquote{img}} (\sphinxstyleliteralemphasis{\sphinxupquote{numpy array}}) \textendash{} A numpy array representing the image.

\item {} 
\sphinxstyleliteralstrong{\sphinxupquote{subsets}} (\sphinxstyleliteralemphasis{\sphinxupquote{list}}) \textendash{} A list of numpy array of each subset.

\end{itemize}

\item[{Returns}] \leavevmode


\item[{Return type}] \leavevmode
None

\end{description}\end{quote}

\end{fulllineitems}

\index{get\_data() (in module readpyne.io)@\spxentry{get\_data()}\spxextra{in module readpyne.io}}

\begin{fulllineitems}
\phantomsection\label{\detokenize{api:readpyne.io.get_data}}\pysiglinewithargsret{\sphinxcode{\sphinxupquote{readpyne.io.}}\sphinxbfcode{\sphinxupquote{get\_data}}}{\emph{folder='data/training'}, \emph{expr='**/Tesco*\_600dpi.j*'}}{}
Load all images in a folder that fit a certain regular expression.
\begin{quote}\begin{description}
\item[{Parameters}] \leavevmode\begin{itemize}
\item {} 
\sphinxstyleliteralstrong{\sphinxupquote{folder}} (\sphinxstyleliteralemphasis{\sphinxupquote{str}}) \textendash{} String path to where the images are.

\item {} 
\sphinxstyleliteralstrong{\sphinxupquote{expr}} (\sphinxstyleliteralemphasis{\sphinxupquote{str}}) \textendash{} A regular expression that will be used to filter images.

\end{itemize}

\item[{Returns}] \leavevmode
A map object of all the images.

\item[{Return type}] \leavevmode
map

\end{description}\end{quote}

\end{fulllineitems}

\index{load\_model() (in module readpyne.io)@\spxentry{load\_model()}\spxextra{in module readpyne.io}}

\begin{fulllineitems}
\phantomsection\label{\detokenize{api:readpyne.io.load_model}}\pysiglinewithargsret{\sphinxcode{\sphinxupquote{readpyne.io.}}\sphinxbfcode{\sphinxupquote{load\_model}}}{\emph{path}}{}
Load a model from a string path.
\begin{quote}\begin{description}
\item[{Parameters}] \leavevmode
\sphinxstyleliteralstrong{\sphinxupquote{path}} (\sphinxstyleliteralemphasis{\sphinxupquote{str}}) \textendash{} A path to the model

\item[{Returns}] \leavevmode


\item[{Return type}] \leavevmode
sklearn model

\end{description}\end{quote}

\end{fulllineitems}

\index{save\_images() (in module readpyne.io)@\spxentry{save\_images()}\spxextra{in module readpyne.io}}

\begin{fulllineitems}
\phantomsection\label{\detokenize{api:readpyne.io.save_images}}\pysiglinewithargsret{\sphinxcode{\sphinxupquote{readpyne.io.}}\sphinxbfcode{\sphinxupquote{save\_images}}}{\emph{image\_list}, \emph{path='outputs/training'}}{}
A list of images exports to a given folder.
\begin{quote}\begin{description}
\item[{Parameters}] \leavevmode\begin{itemize}
\item {} 
\sphinxstyleliteralstrong{\sphinxupquote{image\_list}} (\sphinxstyleliteralemphasis{\sphinxupquote{list}}) \textendash{} A list of images represented as numpy.arrays

\item {} 
\sphinxstyleliteralstrong{\sphinxupquote{path}} (\sphinxstyleliteralemphasis{\sphinxupquote{string}}) \textendash{} A path to the folder.

\end{itemize}

\item[{Returns}] \leavevmode


\item[{Return type}] \leavevmode
None

\end{description}\end{quote}

\end{fulllineitems}

\index{save\_stack() (in module readpyne.io)@\spxentry{save\_stack()}\spxextra{in module readpyne.io}}

\begin{fulllineitems}
\phantomsection\label{\detokenize{api:readpyne.io.save_stack}}\pysiglinewithargsret{\sphinxcode{\sphinxupquote{readpyne.io.}}\sphinxbfcode{\sphinxupquote{save\_stack}}}{\emph{subs}, \emph{features}, \emph{folder}}{}
Get subsets and features and export them.
\begin{quote}\begin{description}
\item[{Parameters}] \leavevmode\begin{itemize}
\item {} 
\sphinxstyleliteralstrong{\sphinxupquote{subs}} (\sphinxstyleliteralemphasis{\sphinxupquote{list}}) \textendash{} A list of numpy arrays representing subsets of the image.

\item {} 
\sphinxstyleliteralstrong{\sphinxupquote{features}} (\sphinxstyleliteralemphasis{\sphinxupquote{list}}) \textendash{} A list of features.

\end{itemize}

\item[{Returns}] \leavevmode
\begin{itemize}
\item {} 
\sphinxstyleemphasis{subs} \textendash{} Same as input

\item {} 
\sphinxstyleemphasis{features} \textendash{} Same as input

\end{itemize}


\end{description}\end{quote}

\end{fulllineitems}

\index{show() (in module readpyne.io)@\spxentry{show()}\spxextra{in module readpyne.io}}

\begin{fulllineitems}
\phantomsection\label{\detokenize{api:readpyne.io.show}}\pysiglinewithargsret{\sphinxcode{\sphinxupquote{readpyne.io.}}\sphinxbfcode{\sphinxupquote{show}}}{\emph{img}}{}
Use the matplotlib pyplot function to show the image.
\begin{quote}\begin{description}
\item[{Parameters}] \leavevmode
\sphinxstyleliteralstrong{\sphinxupquote{img}} (\sphinxstyleliteralemphasis{\sphinxupquote{numpy array}}) \textendash{} A numpy array representing the image.

\item[{Returns}] \leavevmode


\item[{Return type}] \leavevmode
None

\end{description}\end{quote}

\end{fulllineitems}



\chapter{Example usage}
\label{\detokenize{example:example-usage}}\label{\detokenize{example::doc}}
\begin{sphinxVerbatim}[commandchars=\\\{\}]
\PYG{c+c1}{\PYGZsh{} third party}
\PYG{k+kn}{import} \PYG{n+nn}{pandas} \PYG{k+kn}{as} \PYG{n+nn}{pd}

\PYG{c+c1}{\PYGZsh{} project}
\PYG{k+kn}{import} \PYG{n+nn}{readpyne.model} \PYG{k+kn}{as} \PYG{n+nn}{m}

\PYG{c+c1}{\PYGZsh{} config}
\PYG{k+kn}{from} \PYG{n+nn}{readpyne.defaults} \PYG{k+kn}{import} \PYG{n}{config}

\PYG{c+c1}{\PYGZsh{} example of getting data}
\PYG{n}{data} \PYG{o}{=} \PYG{n}{m}\PYG{o}{.}\PYG{n}{make\PYGZus{}training\PYGZus{}data}\PYG{p}{(}
    \PYG{n}{config}\PYG{p}{[}\PYG{l+s+s1}{\PYGZsq{}}\PYG{l+s+s1}{paths}\PYG{l+s+s1}{\PYGZsq{}}\PYG{p}{]}\PYG{p}{[}\PYG{l+s+s1}{\PYGZsq{}}\PYG{l+s+s1}{training\PYGZus{}data\PYGZus{}folder}\PYG{l+s+s1}{\PYGZsq{}}\PYG{p}{]}\PYG{p}{,}
    \PYG{n}{config}\PYG{p}{[}\PYG{l+s+s1}{\PYGZsq{}}\PYG{l+s+s1}{training\PYGZus{}image\PYGZus{}re}\PYG{l+s+s1}{\PYGZsq{}}\PYG{p}{]}\PYG{p}{,}
    \PYG{n}{config}\PYG{p}{[}\PYG{l+s+s1}{\PYGZsq{}}\PYG{l+s+s1}{paths}\PYG{l+s+s1}{\PYGZsq{}}\PYG{p}{]}\PYG{p}{[}\PYG{l+s+s1}{\PYGZsq{}}\PYG{l+s+s1}{training\PYGZus{}output\PYGZus{}folder}\PYG{l+s+s1}{\PYGZsq{}}\PYG{p}{]}\PYG{p}{,}
    \PYG{n+nb+bp}{True}
\PYG{p}{)}

\PYG{c+c1}{\PYGZsh{} example of taking prelabelled data and training a model}
\PYG{n}{training\PYGZus{}data} \PYG{o}{=} \PYG{n}{pd}\PYG{o}{.}\PYG{n}{read\PYGZus{}csv}\PYG{p}{(}\PYG{n}{config}\PYG{p}{[}\PYG{l+s+s1}{\PYGZsq{}}\PYG{l+s+s1}{paths}\PYG{l+s+s1}{\PYGZsq{}}\PYG{p}{]}\PYG{p}{[}\PYG{l+s+s1}{\PYGZsq{}}\PYG{l+s+s1}{training\PYGZus{}data}\PYG{l+s+s1}{\PYGZsq{}}\PYG{p}{]}\PYG{p}{)}

\PYG{n}{model}\PYG{p}{,} \PYG{p}{(}\PYG{n}{X\PYGZus{}test}\PYG{p}{,} \PYG{n}{y\PYGZus{}test}\PYG{p}{)} \PYG{o}{=} \PYG{n}{m}\PYG{o}{.}\PYG{n}{train\PYGZus{}model}\PYG{p}{(}
    \PYG{n}{training\PYGZus{}data}\PYG{o}{.}\PYG{n}{iloc}\PYG{p}{[}\PYG{p}{:}\PYG{p}{,}\PYG{l+m+mi}{1}\PYG{p}{:}\PYG{p}{]}\PYG{p}{,}
    \PYG{n}{training\PYGZus{}data}\PYG{o}{.}\PYG{n}{iloc}\PYG{p}{[}\PYG{p}{:}\PYG{p}{,}\PYG{l+m+mi}{0} \PYG{p}{]}\PYG{p}{,}
    \PYG{n}{report}\PYG{o}{=}\PYG{n+nb+bp}{True}\PYG{p}{,}
    \PYG{n}{save\PYGZus{}path}\PYG{o}{=}\PYG{n}{config}\PYG{p}{[}\PYG{l+s+s1}{\PYGZsq{}}\PYG{l+s+s1}{model}\PYG{l+s+s1}{\PYGZsq{}}\PYG{p}{]}\PYG{p}{[}\PYG{l+s+s1}{\PYGZsq{}}\PYG{l+s+s1}{classifier}\PYG{l+s+s1}{\PYGZsq{}}\PYG{p}{]}
\PYG{p}{)}

\PYG{c+c1}{\PYGZsh{} example of extracting data from all images in a folder}
\PYG{n}{m}\PYG{o}{.}\PYG{n}{extract}\PYG{p}{(}\PYG{n}{config}\PYG{p}{[}\PYG{l+s+s1}{\PYGZsq{}}\PYG{l+s+s1}{paths}\PYG{l+s+s1}{\PYGZsq{}}\PYG{p}{]}\PYG{p}{[}\PYG{l+s+s1}{\PYGZsq{}}\PYG{l+s+s1}{classifier\PYGZus{}input}\PYG{l+s+s1}{\PYGZsq{}}\PYG{p}{]}\PYG{p}{,}
          \PYG{n}{config}\PYG{p}{[}\PYG{l+s+s1}{\PYGZsq{}}\PYG{l+s+s1}{model}\PYG{l+s+s1}{\PYGZsq{}}\PYG{p}{]}\PYG{p}{[}\PYG{l+s+s1}{\PYGZsq{}}\PYG{l+s+s1}{classifier}\PYG{l+s+s1}{\PYGZsq{}}\PYG{p}{]}\PYG{p}{,}
          \PYG{n}{config}\PYG{p}{[}\PYG{l+s+s1}{\PYGZsq{}}\PYG{l+s+s1}{paths}\PYG{l+s+s1}{\PYGZsq{}}\PYG{p}{]}\PYG{p}{[}\PYG{l+s+s1}{\PYGZsq{}}\PYG{l+s+s1}{classifier\PYGZus{}output}\PYG{l+s+s1}{\PYGZsq{}}\PYG{p}{]}\PYG{p}{)}
\end{sphinxVerbatim}


\renewcommand{\indexname}{Python Module Index}
\begin{sphinxtheindex}
\let\bigletter\sphinxstyleindexlettergroup
\bigletter{r}
\item\relax\sphinxstyleindexentry{readpyne.core}\sphinxstyleindexpageref{api:\detokenize{module-readpyne.core}}
\item\relax\sphinxstyleindexentry{readpyne.decorators}\sphinxstyleindexpageref{api:\detokenize{module-readpyne.decorators}}
\item\relax\sphinxstyleindexentry{readpyne.io}\sphinxstyleindexpageref{api:\detokenize{module-readpyne.io}}
\item\relax\sphinxstyleindexentry{readpyne.model}\sphinxstyleindexpageref{api:\detokenize{module-readpyne.model}}
\end{sphinxtheindex}

\renewcommand{\indexname}{Index}
\printindex
\end{document}